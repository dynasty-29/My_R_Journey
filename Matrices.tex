% Options for packages loaded elsewhere
\PassOptionsToPackage{unicode}{hyperref}
\PassOptionsToPackage{hyphens}{url}
%
\documentclass[
]{article}
\usepackage{amsmath,amssymb}
\usepackage{lmodern}
\usepackage{iftex}
\ifPDFTeX
  \usepackage[T1]{fontenc}
  \usepackage[utf8]{inputenc}
  \usepackage{textcomp} % provide euro and other symbols
\else % if luatex or xetex
  \usepackage{unicode-math}
  \defaultfontfeatures{Scale=MatchLowercase}
  \defaultfontfeatures[\rmfamily]{Ligatures=TeX,Scale=1}
\fi
% Use upquote if available, for straight quotes in verbatim environments
\IfFileExists{upquote.sty}{\usepackage{upquote}}{}
\IfFileExists{microtype.sty}{% use microtype if available
  \usepackage[]{microtype}
  \UseMicrotypeSet[protrusion]{basicmath} % disable protrusion for tt fonts
}{}
\makeatletter
\@ifundefined{KOMAClassName}{% if non-KOMA class
  \IfFileExists{parskip.sty}{%
    \usepackage{parskip}
  }{% else
    \setlength{\parindent}{0pt}
    \setlength{\parskip}{6pt plus 2pt minus 1pt}}
}{% if KOMA class
  \KOMAoptions{parskip=half}}
\makeatother
\usepackage{xcolor}
\usepackage[margin=1in]{geometry}
\usepackage{color}
\usepackage{fancyvrb}
\newcommand{\VerbBar}{|}
\newcommand{\VERB}{\Verb[commandchars=\\\{\}]}
\DefineVerbatimEnvironment{Highlighting}{Verbatim}{commandchars=\\\{\}}
% Add ',fontsize=\small' for more characters per line
\usepackage{framed}
\definecolor{shadecolor}{RGB}{248,248,248}
\newenvironment{Shaded}{\begin{snugshade}}{\end{snugshade}}
\newcommand{\AlertTok}[1]{\textcolor[rgb]{0.94,0.16,0.16}{#1}}
\newcommand{\AnnotationTok}[1]{\textcolor[rgb]{0.56,0.35,0.01}{\textbf{\textit{#1}}}}
\newcommand{\AttributeTok}[1]{\textcolor[rgb]{0.77,0.63,0.00}{#1}}
\newcommand{\BaseNTok}[1]{\textcolor[rgb]{0.00,0.00,0.81}{#1}}
\newcommand{\BuiltInTok}[1]{#1}
\newcommand{\CharTok}[1]{\textcolor[rgb]{0.31,0.60,0.02}{#1}}
\newcommand{\CommentTok}[1]{\textcolor[rgb]{0.56,0.35,0.01}{\textit{#1}}}
\newcommand{\CommentVarTok}[1]{\textcolor[rgb]{0.56,0.35,0.01}{\textbf{\textit{#1}}}}
\newcommand{\ConstantTok}[1]{\textcolor[rgb]{0.00,0.00,0.00}{#1}}
\newcommand{\ControlFlowTok}[1]{\textcolor[rgb]{0.13,0.29,0.53}{\textbf{#1}}}
\newcommand{\DataTypeTok}[1]{\textcolor[rgb]{0.13,0.29,0.53}{#1}}
\newcommand{\DecValTok}[1]{\textcolor[rgb]{0.00,0.00,0.81}{#1}}
\newcommand{\DocumentationTok}[1]{\textcolor[rgb]{0.56,0.35,0.01}{\textbf{\textit{#1}}}}
\newcommand{\ErrorTok}[1]{\textcolor[rgb]{0.64,0.00,0.00}{\textbf{#1}}}
\newcommand{\ExtensionTok}[1]{#1}
\newcommand{\FloatTok}[1]{\textcolor[rgb]{0.00,0.00,0.81}{#1}}
\newcommand{\FunctionTok}[1]{\textcolor[rgb]{0.00,0.00,0.00}{#1}}
\newcommand{\ImportTok}[1]{#1}
\newcommand{\InformationTok}[1]{\textcolor[rgb]{0.56,0.35,0.01}{\textbf{\textit{#1}}}}
\newcommand{\KeywordTok}[1]{\textcolor[rgb]{0.13,0.29,0.53}{\textbf{#1}}}
\newcommand{\NormalTok}[1]{#1}
\newcommand{\OperatorTok}[1]{\textcolor[rgb]{0.81,0.36,0.00}{\textbf{#1}}}
\newcommand{\OtherTok}[1]{\textcolor[rgb]{0.56,0.35,0.01}{#1}}
\newcommand{\PreprocessorTok}[1]{\textcolor[rgb]{0.56,0.35,0.01}{\textit{#1}}}
\newcommand{\RegionMarkerTok}[1]{#1}
\newcommand{\SpecialCharTok}[1]{\textcolor[rgb]{0.00,0.00,0.00}{#1}}
\newcommand{\SpecialStringTok}[1]{\textcolor[rgb]{0.31,0.60,0.02}{#1}}
\newcommand{\StringTok}[1]{\textcolor[rgb]{0.31,0.60,0.02}{#1}}
\newcommand{\VariableTok}[1]{\textcolor[rgb]{0.00,0.00,0.00}{#1}}
\newcommand{\VerbatimStringTok}[1]{\textcolor[rgb]{0.31,0.60,0.02}{#1}}
\newcommand{\WarningTok}[1]{\textcolor[rgb]{0.56,0.35,0.01}{\textbf{\textit{#1}}}}
\usepackage{graphicx}
\makeatletter
\def\maxwidth{\ifdim\Gin@nat@width>\linewidth\linewidth\else\Gin@nat@width\fi}
\def\maxheight{\ifdim\Gin@nat@height>\textheight\textheight\else\Gin@nat@height\fi}
\makeatother
% Scale images if necessary, so that they will not overflow the page
% margins by default, and it is still possible to overwrite the defaults
% using explicit options in \includegraphics[width, height, ...]{}
\setkeys{Gin}{width=\maxwidth,height=\maxheight,keepaspectratio}
% Set default figure placement to htbp
\makeatletter
\def\fps@figure{htbp}
\makeatother
\setlength{\emergencystretch}{3em} % prevent overfull lines
\providecommand{\tightlist}{%
  \setlength{\itemsep}{0pt}\setlength{\parskip}{0pt}}
\setcounter{secnumdepth}{-\maxdimen} % remove section numbering
\ifLuaTeX
  \usepackage{selnolig}  % disable illegal ligatures
\fi
\IfFileExists{bookmark.sty}{\usepackage{bookmark}}{\usepackage{hyperref}}
\IfFileExists{xurl.sty}{\usepackage{xurl}}{} % add URL line breaks if available
\urlstyle{same} % disable monospaced font for URLs
\hypersetup{
  pdftitle={Matrices},
  pdfauthor={Dynasty},
  hidelinks,
  pdfcreator={LaTeX via pandoc}}

\title{Matrices}
\author{Dynasty}
\date{2022-07-12}

\begin{document}
\maketitle

\hypertarget{matrices}{%
\section{Matrices}\label{matrices}}

A matrix is a two dimensional data structure that is similar to a vector
but additionally contains the dimension attribute.

\hypertarget{creating}{%
\subsection{1. Creating}\label{creating}}

When creating a matrix, the last two arguments to matrix tell it the
number of rows and columns the matrix should have.

\hypertarget{matrix-code-example}{%
\paragraph{1.1 Matrix Code Example}\label{matrix-code-example}}

You can use the byrow=TRUE argument to create a matrix by rows instead
of by columns as shown below:

\hypertarget{lets-create-a-matrix-mymat}{%
\subparagraph{Let's create a matrix
mymat}\label{lets-create-a-matrix-mymat}}

\begin{Shaded}
\begin{Highlighting}[]
\NormalTok{mymat }\OtherTok{\textless{}{-}} \FunctionTok{matrix}\NormalTok{(}\DecValTok{1}\SpecialCharTok{:}\DecValTok{12}\NormalTok{,}\DecValTok{4}\NormalTok{,}\DecValTok{3}\NormalTok{)}
\end{Highlighting}
\end{Shaded}

\hypertarget{printing-out-matrix}{%
\subparagraph{printing out matrix}\label{printing-out-matrix}}

\begin{Shaded}
\begin{Highlighting}[]
\NormalTok{mymat}
\end{Highlighting}
\end{Shaded}

\begin{verbatim}
##      [,1] [,2] [,3]
## [1,]    1    5    9
## [2,]    2    6   10
## [3,]    3    7   11
## [4,]    4    8   12
\end{verbatim}

\hypertarget{matrix-code-example-1}{%
\paragraph{1.2 Matrix Code Example}\label{matrix-code-example-1}}

\hypertarget{lets-use-the-byrowtrue-argument-to-create-a-matrix-by-rows-instead-of-by-columns-as-shown-below}{%
\subparagraph{Let's use the byrow=TRUE argument to create a matrix by
rows instead of by columns as shown
below}\label{lets-use-the-byrowtrue-argument-to-create-a-matrix-by-rows-instead-of-by-columns-as-shown-below}}

\begin{Shaded}
\begin{Highlighting}[]
\NormalTok{mymat }\OtherTok{\textless{}{-}} \FunctionTok{matrix}\NormalTok{(}\DecValTok{1}\SpecialCharTok{:}\DecValTok{12}\NormalTok{,}\AttributeTok{ncol=}\DecValTok{3}\NormalTok{,}\AttributeTok{byrow=}\ConstantTok{TRUE}\NormalTok{)}
\end{Highlighting}
\end{Shaded}

\hypertarget{print-our-matrix}{%
\subparagraph{print our matrix}\label{print-our-matrix}}

\begin{Shaded}
\begin{Highlighting}[]
\NormalTok{mymat}
\end{Highlighting}
\end{Shaded}

\begin{verbatim}
##      [,1] [,2] [,3]
## [1,]    1    2    3
## [2,]    4    5    6
## [3,]    7    8    9
## [4,]   10   11   12
\end{verbatim}

\hypertarget{naming}{%
\subsection{2. Naming}\label{naming}}

\hypertarget{matrix-naming-code-example}{%
\paragraph{2.1 Matrix Naming Code
Example}\label{matrix-naming-code-example}}

\hypertarget{in-order-to-remember-what-is-stored-in-a-matrix-you-can-add-the-names-of-the-columns-and-rows.-this-will-also-help-you-to-read-the-data-as-well-as-select-elements-from-the-matrix.}{%
\paragraph{\# In order to remember what is stored in a matrix, you can
add the names of the columns and rows. This will also help you to read
the data as well as select elements from the
matrix.}\label{in-order-to-remember-what-is-stored-in-a-matrix-you-can-add-the-names-of-the-columns-and-rows.-this-will-also-help-you-to-read-the-data-as-well-as-select-elements-from-the-matrix.}}

\hypertarget{lets-create-our-matrix}{%
\paragraph{Let's create our matrix}\label{lets-create-our-matrix}}

\begin{Shaded}
\begin{Highlighting}[]
\NormalTok{kenya }\OtherTok{\textless{}{-}} \FunctionTok{c}\NormalTok{(}\FloatTok{460.998}\NormalTok{, }\FloatTok{314.4}\NormalTok{) }
\NormalTok{ethiopia }\OtherTok{\textless{}{-}} \FunctionTok{c}\NormalTok{(}\FloatTok{290.475}\NormalTok{, }\FloatTok{247.900}\NormalTok{) }
\NormalTok{chad }\OtherTok{\textless{}{-}} \FunctionTok{c}\NormalTok{(}\FloatTok{309.306}\NormalTok{, }\FloatTok{165.8}\NormalTok{)}
\end{Highlighting}
\end{Shaded}

\hypertarget{then-create-a-matrix-geography-matrix}{%
\paragraph{Then create a matrix geography
matrix}\label{then-create-a-matrix-geography-matrix}}

\begin{Shaded}
\begin{Highlighting}[]
\NormalTok{geography\_matrix }\OtherTok{\textless{}{-}} \FunctionTok{matrix}\NormalTok{(}\FunctionTok{c}\NormalTok{(kenya, ethiopia, chad), }\AttributeTok{nrow =} \DecValTok{3}\NormalTok{, }\AttributeTok{byrow =} \ConstantTok{TRUE}\NormalTok{)}
\end{Highlighting}
\end{Shaded}

\hypertarget{now-lets-start-naming}{%
\subparagraph{Now let's start naming}\label{now-lets-start-naming}}

\begin{Shaded}
\begin{Highlighting}[]
\NormalTok{location }\OtherTok{\textless{}{-}} \FunctionTok{c}\NormalTok{(}\StringTok{"Lat"}\NormalTok{, }\StringTok{"Long"}\NormalTok{)}
\NormalTok{countries }\OtherTok{\textless{}{-}} \FunctionTok{c}\NormalTok{(}\StringTok{"Kenya"}\NormalTok{, }\StringTok{"Ethiopia"}\NormalTok{, }\StringTok{"Chad"}\NormalTok{)}
\end{Highlighting}
\end{Shaded}

\hypertarget{now-lets-name-the-columns-with-location}{%
\subparagraph{Now let's name the columns with
location}\label{now-lets-name-the-columns-with-location}}

\begin{Shaded}
\begin{Highlighting}[]
\FunctionTok{colnames}\NormalTok{(geography\_matrix) }\OtherTok{\textless{}{-}}\NormalTok{ location}
\end{Highlighting}
\end{Shaded}

\hypertarget{lets-name-the-rows-with-countries}{%
\subsubsection{Let's name the rows with
countries}\label{lets-name-the-rows-with-countries}}

\begin{Shaded}
\begin{Highlighting}[]
\FunctionTok{rownames}\NormalTok{(geography\_matrix) }\OtherTok{\textless{}{-}}\NormalTok{ countries}
\end{Highlighting}
\end{Shaded}

\hypertarget{now-lets-print-final-product}{%
\subparagraph{Now let's print final
product}\label{now-lets-print-final-product}}

\begin{Shaded}
\begin{Highlighting}[]
\NormalTok{geography\_matrix}
\end{Highlighting}
\end{Shaded}

\begin{verbatim}
##              Lat  Long
## Kenya    460.998 314.4
## Ethiopia 290.475 247.9
## Chad     309.306 165.8
\end{verbatim}

\hypertarget{now-lets-create-a-matrix-family-with-column-names-name-age-gender-and-occupation.}{%
\paragraph{Now let's create a matrix family with column names Name, Age,
Gender and
Occupation.}\label{now-lets-create-a-matrix-family-with-column-names-name-age-gender-and-occupation.}}

\begin{Shaded}
\begin{Highlighting}[]
\NormalTok{Matriachy }\OtherTok{\textless{}{-}} \FunctionTok{c}\NormalTok{(}\StringTok{"Matriachy"}\NormalTok{, }\DecValTok{60}\NormalTok{, }\StringTok{"F"}\NormalTok{, }\StringTok{"Queen"}\NormalTok{)}
\NormalTok{Ryan }\OtherTok{\textless{}{-}} \FunctionTok{c}\NormalTok{(}\StringTok{"Ryan"}\NormalTok{, }\DecValTok{7}\NormalTok{, }\StringTok{"M"}\NormalTok{, }\StringTok{"Student"}\NormalTok{)}
\NormalTok{Dynasty }\OtherTok{\textless{}{-}} \FunctionTok{c}\NormalTok{(}\StringTok{"Dynasty"}\NormalTok{, }\DecValTok{30}\NormalTok{, }\StringTok{"F"}\NormalTok{, }\StringTok{"Scientist"}\NormalTok{)}
\NormalTok{Migan }\OtherTok{\textless{}{-}} \FunctionTok{c}\NormalTok{(}\StringTok{"Migan"}\NormalTok{, }\DecValTok{26}\NormalTok{, }\StringTok{"F"}\NormalTok{, }\StringTok{"Chef"}\NormalTok{)}
\NormalTok{Wolfcode }\OtherTok{\textless{}{-}} \FunctionTok{c}\NormalTok{(}\StringTok{"Wolfcode"}\NormalTok{, }\DecValTok{19}\NormalTok{, }\StringTok{"M"}\NormalTok{, }\StringTok{"Engineer"}\NormalTok{)}
\end{Highlighting}
\end{Shaded}

\begin{Shaded}
\begin{Highlighting}[]
\NormalTok{family }\OtherTok{\textless{}{-}} \FunctionTok{matrix}\NormalTok{(}\FunctionTok{c}\NormalTok{(Matriachy, Ryan, Dynasty, Migan, Wolfcode), }\AttributeTok{nrow =} \DecValTok{5}\NormalTok{, }\AttributeTok{byrow =} \ConstantTok{TRUE}\NormalTok{)}
\end{Highlighting}
\end{Shaded}

\begin{Shaded}
\begin{Highlighting}[]
\NormalTok{Header }\OtherTok{\textless{}{-}} \FunctionTok{c}\NormalTok{(}\StringTok{"Name"}\NormalTok{, }\StringTok{"Age"}\NormalTok{, }\StringTok{"Gender"}\NormalTok{, }\StringTok{"Occupation"}\NormalTok{)}
\end{Highlighting}
\end{Shaded}

\begin{Shaded}
\begin{Highlighting}[]
\NormalTok{Name }\OtherTok{\textless{}{-}} \FunctionTok{c}\NormalTok{(}\StringTok{"Matriachy"}\NormalTok{, }\StringTok{"Ryan"}\NormalTok{, }\StringTok{"Dynasty"}\NormalTok{, }\StringTok{"Migan"}\NormalTok{, }\StringTok{"Wolfcode"}\NormalTok{)}
\end{Highlighting}
\end{Shaded}

\begin{Shaded}
\begin{Highlighting}[]
\FunctionTok{colnames}\NormalTok{(family) }\OtherTok{\textless{}{-}}\NormalTok{ Header}
\end{Highlighting}
\end{Shaded}

\begin{Shaded}
\begin{Highlighting}[]
\FunctionTok{rownames}\NormalTok{(family) }\OtherTok{\textless{}{-}}\NormalTok{ Name}
\end{Highlighting}
\end{Shaded}

\begin{Shaded}
\begin{Highlighting}[]
\NormalTok{family}
\end{Highlighting}
\end{Shaded}

\begin{verbatim}
##           Name        Age  Gender Occupation 
## Matriachy "Matriachy" "60" "F"    "Queen"    
## Ryan      "Ryan"      "7"  "M"    "Student"  
## Dynasty   "Dynasty"   "30" "F"    "Scientist"
## Migan     "Migan"     "26" "F"    "Chef"     
## Wolfcode  "Wolfcode"  "19" "M"    "Engineer"
\end{verbatim}

\hypertarget{adding-a-column}{%
\subsection{3. Adding a Column}\label{adding-a-column}}

You can add a row to a matrix using the rbind() function.

\hypertarget{adding-a-column-code-example}{%
\paragraph{3.1 Adding a Column Code
Example}\label{adding-a-column-code-example}}

\hypertarget{lets-create-an-x-matrix}{%
\paragraph{let's create an x matrix}\label{lets-create-an-x-matrix}}

\begin{Shaded}
\begin{Highlighting}[]
\NormalTok{x }\OtherTok{\textless{}{-}} \FunctionTok{matrix}\NormalTok{(}\DecValTok{1}\SpecialCharTok{:}\DecValTok{9}\NormalTok{, }\AttributeTok{nrow =} \DecValTok{3}\NormalTok{)}
\end{Highlighting}
\end{Shaded}

\hypertarget{then-ass-a-column}{%
\paragraph{Then ass a column}\label{then-ass-a-column}}

\begin{Shaded}
\begin{Highlighting}[]
\FunctionTok{cbind}\NormalTok{(x, }\FunctionTok{c}\NormalTok{(}\DecValTok{1}\NormalTok{, }\DecValTok{2}\NormalTok{, }\DecValTok{3}\NormalTok{))}
\end{Highlighting}
\end{Shaded}

\begin{verbatim}
##      [,1] [,2] [,3] [,4]
## [1,]    1    4    7    1
## [2,]    2    5    8    2
## [3,]    3    6    9    3
\end{verbatim}

\hypertarget{lets-add-a-column-residence-to-your-fictional-family-matrix}{%
\subparagraph{Let's Add a column residence to your fictional family
matrix}\label{lets-add-a-column-residence-to-your-fictional-family-matrix}}

\begin{Shaded}
\begin{Highlighting}[]
\FunctionTok{cbind}\NormalTok{(family, }\FunctionTok{c}\NormalTok{(}\StringTok{"Arabia Terra"}\NormalTok{, }\StringTok{"Amazonis Planitia"}\NormalTok{, }\StringTok{"Amazonis Planitia"}\NormalTok{, }\StringTok{"Arabia Terra"}\NormalTok{, }\StringTok{"Amazonis Planitia"}\NormalTok{))}
\end{Highlighting}
\end{Shaded}

\begin{verbatim}
##           Name        Age  Gender Occupation                     
## Matriachy "Matriachy" "60" "F"    "Queen"     "Arabia Terra"     
## Ryan      "Ryan"      "7"  "M"    "Student"   "Amazonis Planitia"
## Dynasty   "Dynasty"   "30" "F"    "Scientist" "Amazonis Planitia"
## Migan     "Migan"     "26" "F"    "Chef"      "Arabia Terra"     
## Wolfcode  "Wolfcode"  "19" "M"    "Engineer"  "Amazonis Planitia"
\end{verbatim}

\hypertarget{adding-a-row}{%
\subsection{4. Adding a Row}\label{adding-a-row}}

\hypertarget{adding-a-row-code-example}{%
\paragraph{4.1 Adding a Row Code
Example}\label{adding-a-row-code-example}}

\hypertarget{lets-create-a-matrix}{%
\paragraph{Let's create a matrix}\label{lets-create-a-matrix}}

\begin{Shaded}
\begin{Highlighting}[]
\NormalTok{x }\OtherTok{\textless{}{-}} \FunctionTok{matrix}\NormalTok{(}\DecValTok{1}\SpecialCharTok{:}\DecValTok{9}\NormalTok{, }\AttributeTok{nrow =} \DecValTok{3}\NormalTok{)}
\end{Highlighting}
\end{Shaded}

\hypertarget{now-lets-add-a-row-using-cbind}{%
\paragraph{now let's add a row using
cbind}\label{now-lets-add-a-row-using-cbind}}

\begin{Shaded}
\begin{Highlighting}[]
\FunctionTok{rbind}\NormalTok{(x,}\FunctionTok{c}\NormalTok{(}\DecValTok{1}\NormalTok{,}\DecValTok{2}\NormalTok{,}\DecValTok{3}\NormalTok{))}
\end{Highlighting}
\end{Shaded}

\begin{verbatim}
##      [,1] [,2] [,3]
## [1,]    1    4    7
## [2,]    2    5    8
## [3,]    3    6    9
## [4,]    1    2    3
\end{verbatim}

\hypertarget{lets-add-a-fictional-character-to-our-fictional-family-matrix}{%
\paragraph{Let's Add a fictional character to our fictional family
matrix}\label{lets-add-a-fictional-character-to-our-fictional-family-matrix}}

\begin{Shaded}
\begin{Highlighting}[]
\FunctionTok{rbind}\NormalTok{(family, }\FunctionTok{c}\NormalTok{(}\StringTok{"Ruby"}\NormalTok{, }\DecValTok{3}\NormalTok{, }\StringTok{"F"}\NormalTok{, }\StringTok{"Youngling"}\NormalTok{, }\StringTok{"Amazonis Planitia"}\NormalTok{))}
\end{Highlighting}
\end{Shaded}

\begin{verbatim}
## Warning in rbind(family, c("Ruby", 3, "F", "Youngling", "Amazonis Planitia")):
## number of columns of result is not a multiple of vector length (arg 2)
\end{verbatim}

\begin{verbatim}
##           Name        Age  Gender Occupation 
## Matriachy "Matriachy" "60" "F"    "Queen"    
## Ryan      "Ryan"      "7"  "M"    "Student"  
## Dynasty   "Dynasty"   "30" "F"    "Scientist"
## Migan     "Migan"     "26" "F"    "Chef"     
## Wolfcode  "Wolfcode"  "19" "M"    "Engineer" 
##           "Ruby"      "3"  "F"    "Youngling"
\end{verbatim}

\end{document}
